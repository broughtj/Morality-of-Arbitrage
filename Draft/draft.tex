\documentclass[11pt,]{article}
\usepackage[left=1in,top=1in,right=1in,bottom=1in]{geometry}
\newcommand*{\authorfont}{\fontfamily{phv}\selectfont}
\usepackage[]{mathpazo}


  \usepackage[T1]{fontenc}
  \usepackage[utf8]{inputenc}



\usepackage{abstract}
\renewcommand{\abstractname}{}    % clear the title
\renewcommand{\absnamepos}{empty} % originally center

\renewenvironment{abstract}
 {{%
    \setlength{\leftmargin}{0mm}
    \setlength{\rightmargin}{\leftmargin}%
  }%
  \relax}
 {\endlist}

\makeatletter
\def\@maketitle{%
  \newpage
%  \null
%  \vskip 2em%
%  \begin{center}%
  \let \footnote \thanks
    {\fontsize{18}{20}\selectfont\raggedright  \setlength{\parindent}{0pt} \@title \par}%
}
%\fi
\makeatother




\setcounter{secnumdepth}{0}



\title{The Morality of Arbitrage  }



\author{\Large Tyler J. Brough\vspace{0.05in} \newline\normalsize\emph{Utah State University}  }


\date{}

\usepackage{titlesec}

\titleformat*{\section}{\normalsize\bfseries}
\titleformat*{\subsection}{\normalsize\itshape}
\titleformat*{\subsubsection}{\normalsize\itshape}
\titleformat*{\paragraph}{\normalsize\itshape}
\titleformat*{\subparagraph}{\normalsize\itshape}


\usepackage{natbib}
\bibliographystyle{apsr}



\newtheorem{hypothesis}{Hypothesis}
\usepackage{setspace}

\makeatletter
\@ifpackageloaded{hyperref}{}{%
\ifxetex
  \usepackage[setpagesize=false, % page size defined by xetex
              unicode=false, % unicode breaks when used with xetex
              xetex]{hyperref}
\else
  \usepackage[unicode=true]{hyperref}
\fi
}
\@ifpackageloaded{color}{
    \PassOptionsToPackage{usenames,dvipsnames}{color}
}{%
    \usepackage[usenames,dvipsnames]{color}
}
\makeatother
\hypersetup{breaklinks=true,
            bookmarks=true,
            pdfauthor={Tyler J. Brough (Utah State University)},
             pdfkeywords = {arbitrage, Dutch books, morality},  
            pdftitle={The Morality of Arbitrage},
            colorlinks=true,
            citecolor=blue,
            urlcolor=blue,
            linkcolor=magenta,
            pdfborder={0 0 0}}
\urlstyle{same}  % don't use monospace font for urls



\begin{document}
	
% \pagenumbering{arabic}% resets `page` counter to 1 
%
% \maketitle

{% \usefont{T1}{pnc}{m}{n}
\setlength{\parindent}{0pt}
\thispagestyle{plain}
{\fontsize{18}{20}\selectfont\raggedright 
\maketitle  % title \par  

}

{
   \vskip 13.5pt\relax \normalsize\fontsize{11}{12} 
\textbf{\authorfont Tyler J. Brough} \hskip 15pt \emph{\small Utah State University}   

}

}







\begin{abstract}

    \hbox{\vrule height .2pt width 39.14pc}

    \vskip 8.5pt % \small 

\noindent In this paper, I argue that one cannot rationally believe that financial
arbitrage is an inherently immoral pursuit, while simultaneously
believing that the pursuit of academic knowledge is an inherently moral
pursuit. I argue that these two seemingly separate activities are really
one and the same. Further those that do hold such contradictory beliefs
are behaving incoherently and have a Dutch book made against them.


\vskip 8.5pt \noindent \emph{Keywords}: arbitrage, Dutch books, morality \par

    \hbox{\vrule height .2pt width 39.14pc}



\end{abstract}


\vskip 6.5pt

\noindent  \section{Introduction}\label{introduction}

The pursuit of knowledge through academic scholarship rests on a
commonly held moral foundation. The pursuit of truth, which is widely
believed to be the goal of scholarship, is thought to be an inherently
moral activity. At the same time, it is an increasingly widely held
belief that financial market arbitrage, the kind pursued by Wall Street
traders, is an inherently immoral activity. Take for example, the case
of the quantitative proprietary trading firm which serves no clients and
which invests only the capital of its principals. Such a firm makes its
profits, often astonishingly large profits, by buying and selling
financial securities and profiting from speculation. A common line of
reasoning is that such a firm is predatory, parasitic, and even
rapacious and as such cannot possibly be acting morally.

In this paper, I argue that one cannot simultaneously hold the above two
positions without behaving incoherently. The process of scholarship and
financial trading are both examples of finding and exploiting Dutch
books. Dutch books are simply arbitrage by another name. Thus
scholarship can be seen, indeed is most properly seen, as a form of
intellectual arbitrage. Just as traders find mispricings in financial
securities and exploit them for financial profit, scholars find
misunderstandings - mispricings by another name - and exploit them to
establish their scholarly reputation.

\citet{Coase1974}, in what he referred to as the economics of the First
Amendment, makes a similar argument by comparing the ``market for
ideas'' and the more mundane ``market for goods.'' In his analysis he
points out a broad inconsistency in the way these two different markets
are viewed: that the market for goods is inefficient and requires
government regulation to tame it, but that it would even be immoral for
government to seek to regulate the market for ideas. Coase points out
that this view is fundamentally inconsistent:

\begin{quote}
\emph{Consider the question of consumer ignorance which is commonly
thought to be a justification for government intervention. It is hard to
believe that the general public is in a better position to evaluate
competing views on economic and social policy than to choose between
different kinds of food.}
\end{quote}

Coase states further:

\begin{quote}
\emph{Because of the view that a free market in ideas is necessary to
the maintenance of democratic institutions and, I believe, for other
reasons also, intellectuals have shown a tendency to exalt the market
for ideas and to depreciate the market for goods. Such an attitude seems
to me unjustified.}
\end{quote}

I argue, that not only is such a view unjustified but it is irredeemably
incoherent and irrational. In pointing out the tendency of intellectuals
to favor the market for ideas over the market for goods, Coase was
echoing \citet{Hayek1945}, who wrote of ``the relative importance of the
different kinds of knowledge'' and the fact that ``one kind of
knowledge, namely, scientific knowledge, occupies now so prominent a
place in public imagination that we tend to forget that it is not the
only kind that is relevant.'' Indeed, by taking advantage of relative
mispricings among commodities the financial trader benefits his fellow
men by allocating resources more efficiently. But this is hardly viewed
as a noble act, not nearly so noble as scientific discovery or academic
expression. Hayek saw keenly the situation:

\begin{quote}
\emph{It is a curious fact that this sort of knowledge should today be
generally regarded with a kind of contempt, and that someone who by such
knowledge gains an advantage over somebody better equipped with
theoretical or technical knowledge is thought to have acted almost
disreputably. To gain advantage from better knowledge of facilities of
communication or transport is sometimes regarded as almost dishonest,
although it is quite as important that society make use of the best
opportunities in this respect as in using the latest scientific
discoveries.}
\end{quote}

The lesson regarding the relative importance of different kinds of
knowledge that Hayek teaches is a very deep one. Indeed, when this
insight is fully appreciated one begins to see that the use of mundane
kinds of knowledge - Hayek's ``knowledge of time and place'' is
essential. It is not exaggeration to state that if the only knowledge
that society could usefully exploit was that from universities, the
press, literature and science we would all starve to death in very short
order. The financial arbitrageur is a perfect example of Hayek's ``man
on the spot'' and his actions benefit his fellows greatly. Still, when
one thinks of the abstract concept of the pursuit of knowledge, images
of the scientist in the lab or of the poet in the library come more
easily to mind than the trader scanning prices on his bank of monitors.
But for Hayek:

\begin{quote}
\emph{the arbitrageur who gains from local differences of commodity
prices, {[}is{]} performing {[}an{]} eminently useful function based on
special knowledge of circumstances of the fleeting moment not known to
others.}
\end{quote}

The actions of the arbitrageur are indeed eminently useful! Consider the
commodities that we require in our daily lives such as sugar, coffee,
wheat, milk, and oil that come to us more efficiently and at lower cost
than would otherwise be the case if not for the actions of commodity
traders. In addition, because of financial traders retirement is secured
more soundly through investment. And risks are insured against,
eliminated and transferred through financial markets that we would
otherwise have to bear.

Hayek compared the price system to ``a kind of machinery for registering
change'' and as a ``system of telecommunications.'' Similarly,
\citet{Scarf1990} has made an analogy between the market institution and
a massively distributed analog computer that computes values in the form
of prices. Following this line of reasoning, all market processes are
knowledge processes. Therefore there is not sharp distinction between
the ``market for ideas'' and the ``market for goods'', as Coase referred
to them. They are simply both markets and differ only in their relative
kinds of knowledge. \citet{Sowell1996} has made this point quite
succinctly:

\begin{quote}
\emph{While market economies are often thought of as money economies,
they are still more so knowledge economies\ldots{} Economic transactions
are purchases and sales of knowledge.}
\end{quote}

\begin{quote}
\emph{After all, the cavemen had the same natural resources at their
disposal as we have today\ldots{} We are all in the business of buying
and selling knowledge from one another, because we are each so
profoundly ignorant of what it takes to complete the whole process of
which we are part.}
\end{quote}

The link between the market for ideas and the market for goods is
further strengthened by considering the work of the decision theorist
Robert Nau. For example, \citet{NauMcCardle1991} state that the
principle of no-arbitrage, which is fundamental in finance theory, is
the common foundation of decision theory, game theory and competitive
market theory. \citet{Nau1999} refers to this as Arbitrage Choice
Theory, following the Bayesian Dutch book arguments of
\citet{deFinetti1937}. A Dutch book is a situation in which an
individual holds incoherent beliefs (beliefs inconsistent with the basic
axioms of probability), and if forced to bet on them exposes themself to
a guaranteed loss. When market transactions are seen as knowledge
transactions, it becomes clear that a Dutch book is equivalent to
arbitrage - the good old fashioned kind often derided from the world of
finance. This leads to the Dutch book theorem, which the Cambridge
Dictionary of Philosophy (1999) defines as:

\begin{quote}
The proposition that anyone who (a) counts a bet on a proposition p as
fair if the odds correspond to his degree of belief that p is true and
who (b) is willing to make any combination of bets he would regard
individually as fair will be vulnerable to a Dutch book provided his
degrees of belief do not conform tot he axioms of the probability
calculus. Thus, anyone of whom (a) and (b) are true and whos degree of
belief in a disjuntion of two incompatible propositions is not equal to
the sum of his degrees of belief in the two propositions taken
individually would be vulnerable to a Dutch book.
\end{quote}

When a trader makes a profit by simultaneously buying and selling
financial securities to exploit relative mispricings, this arbitrage is
simply the exploitation of a Dutch book expressed as incoherent market
prices. Likewise the intellectual pursuit of knowledge, what Coase
called the ``market for ideas'' can be seen as the pursuit of arbitrage,
or Dutch books, in the existing body of academic knowledge. Consider a
young scholar who discovers an error in the received theory of their
specific academic domain, and who makes a name for themself by
publishing an article correcting the error in a reputable academic
journal. This can now be easily understood to be the exploitation of a
Dutch book, or intellectual arbitrage.

My argument can now be stated simply as follows. One cannot
simultaneously believe: (a) that the pursuit of academic or scientific
knowledge is inherently moral, and (b) that the pursuit of capitalist
profits through financial arbitrage is inherently immoral.

There is thus a logical defeator for the belief that arbitrage is
immoral. Financial arbitrage rests on the same moral foundation as
scholarship - the pursuit of knowledge - albeit in a more plain and
mundate way.

\newpage
\singlespacing 
\bibliography{master.bib}

\end{document}
