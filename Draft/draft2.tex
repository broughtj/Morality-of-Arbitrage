\documentclass[11pt,]{article}
\usepackage[left=1in,top=1in,right=1in,bottom=1in]{geometry}
\newcommand*{\authorfont}{\fontfamily{phv}\selectfont}
\usepackage[]{mathpazo}


  \usepackage[T1]{fontenc}
  \usepackage[utf8]{inputenc}



\usepackage{abstract}
\renewcommand{\abstractname}{}    % clear the title
\renewcommand{\absnamepos}{empty} % originally center

\renewenvironment{abstract}
 {{%
    \setlength{\leftmargin}{0mm}
    \setlength{\rightmargin}{\leftmargin}%
  }%
  \relax}
 {\endlist}

\makeatletter
\def\@maketitle{%
  \newpage
%  \null
%  \vskip 2em%
%  \begin{center}%
  \let \footnote \thanks
    {\fontsize{18}{20}\selectfont\raggedright  \setlength{\parindent}{0pt} \@title \par}%
}
%\fi
\makeatother




\setcounter{secnumdepth}{0}



\title{The Morality of Arbitrage  }



\author{\Large Tyler J. Brough\vspace{0.05in} \newline\normalsize\emph{Utah State University}  }


\date{}

\usepackage{titlesec}

\titleformat*{\section}{\normalsize\bfseries}
\titleformat*{\subsection}{\normalsize\itshape}
\titleformat*{\subsubsection}{\normalsize\itshape}
\titleformat*{\paragraph}{\normalsize\itshape}
\titleformat*{\subparagraph}{\normalsize\itshape}


\usepackage{natbib}
\bibliographystyle{apsr}



\newtheorem{hypothesis}{Hypothesis}
\usepackage{setspace}

\makeatletter
\@ifpackageloaded{hyperref}{}{%
\ifxetex
  \usepackage[setpagesize=false, % page size defined by xetex
              unicode=false, % unicode breaks when used with xetex
              xetex]{hyperref}
\else
  \usepackage[unicode=true]{hyperref}
\fi
}
\@ifpackageloaded{color}{
    \PassOptionsToPackage{usenames,dvipsnames}{color}
}{%
    \usepackage[usenames,dvipsnames]{color}
}
\makeatother
\hypersetup{breaklinks=true,
            bookmarks=true,
            pdfauthor={Tyler J. Brough (Utah State University)},
             pdfkeywords = {arbitrage, Dutch books, morality},  
            pdftitle={The Morality of Arbitrage},
            colorlinks=true,
            citecolor=blue,
            urlcolor=blue,
            linkcolor=magenta,
            pdfborder={0 0 0}}
\urlstyle{same}  % don't use monospace font for urls



\begin{document}
	
% \pagenumbering{arabic}% resets `page` counter to 1 
%
% \maketitle

{% \usefont{T1}{pnc}{m}{n}
\setlength{\parindent}{0pt}
\thispagestyle{plain}
{\fontsize{18}{20}\selectfont\raggedright 
\maketitle  % title \par  

}

{
   \vskip 13.5pt\relax \normalsize\fontsize{11}{12} 
\textbf{\authorfont Tyler J. Brough} \hskip 15pt \emph{\small Utah State University}   

}

}







\begin{abstract}

    \hbox{\vrule height .2pt width 39.14pc}

    \vskip 8.5pt % \small 

\noindent In this paper, I argue that one cannot rationally believe that financial
arbitrage is an inherently immoral pursuit, while simultaneously
believing that the pursuit of academic knowledge is an inherently moral
pursuit. I argue that these two seemingly separate activities are really
one and the same. Further those that do hold such contradictory beliefs
are behaving incoherently and have a Dutch book made against them.


\vskip 8.5pt \noindent \emph{Keywords}: arbitrage, Dutch books, morality \par

    \hbox{\vrule height .2pt width 39.14pc}



\end{abstract}


\vskip 6.5pt

\noindent \doublespacing \begin{quote}
{[}T{]}he shipper who earns his living from using otherwise empty or
half-filled journeys of tramp-steamers, or the estate agent whose whole
knowledge is almost exclusively one of temporary opportunities, or the
\emph{arbitrageur} who gains from local differences of commodity prices,
are all performing eminently useful functions based on special knowledge
of circumstances of the fleeting moment not known to others. -- F.A.
Hayek
\end{quote}

\section{Introduction}\label{introduction}

The pursuit of knowledge through academic scholarship rests on a
commonly held moral foundation. The pursuit of truth, which is widely
believed to be the goal of scholarship, is thought to be an inherently
moral activity. At the same time, it is an increasingly widely held
belief that financial market arbitrage, the kind pursued by Wall Street
traders, is an inherently immoral activity. Take for example, the case
of the quantitative proprietary trading firm which serves no clients and
which invests only the capital of its principals. Such a firm makes its
profits, often astonishingly large profits, by buying and selling
financial securities and profiting from speculation. A common line of
reasoning is that such a firm is predatory, parasitic, and even
rapacious and as such cannot possibly be acting morally.

In this paper, I argue that one cannot simultaneously hold the above two
positions without behaving incoherently. The process of scholarship and
financial trading are both examples of finding and exploiting Dutch
books. Dutch books are simply arbitrage by another name. Thus
scholarship can be seen, indeed is most properly seen, as a form of
intellectual arbitrage. Just as traders find mispricings in financial
securities and exploit them for financial profit, scholars find
misunderstandings - mispricings by another name - and exploit them to
establish their scholarly reputation.

The paper is organized as follows. I outline an original discussion due
to \citet{Coase1974} of the ``market for ideas'' and the ``market for
goods,'' and discuss his conclusion that their different treatment is
incongruous in Section 2. I then discuss the idea, due to
\citet{Hayek1945}, that the economic process is an information process
in its most essential treatment in Section 3. In Section 4 I outline the
concept of subjective probability following \citet{deFinetti1937}, as
well as Arbitrage Choice Theory put forward by \citet{Nau1999}. In
Section 5 I explain the Dutch Book Theorem as essential background to my
argument. I give a few chosen examples of scholarly debates framed as
Dutch Book arguments in Section 6. In Section 7 I present my own Dutch
book argument. Section 8 concludes the paper.

\section{The Market for Goods and the Market for
Ideas}\label{the-market-for-goods-and-the-market-for-ideas}

\citet{Coase1974} points out that there has been a common tradition of
drawing a sharp distinction between the ``market for goods'' and the
``market for ideas.'' The ``market for ideas'' is the domain of
intellectuals and consists of speech, writing, and religious belief -
activities covered by the First Amendment. Coase notes that the common
view has been that the ``market for goods'' requires extensive
regulation to promote public interest, while the ``market for ideas''
would be greatly harmed by such regulatory activity. He writes ``the
Western world, by and large, accepts the distinction and the policy
recommendations that go with it.'' For Coase, such an attitude is
unjustified.

Coase writes that ``\ldots{} belief in a free market in ideas does not
have the same roots as belief in the value of free trade in goods.'' And
further that, ``\ldots{} this view of the peculiar status of the market
for ideas has been nourished by a commitment to democracy as exemplified
in the political institutions of the United States, for whose efficient
working a market in ideas not subject to government regulation is
considered essential.'' Speaking of intellectuals Coase highlights that
``intellectuals have shown a tendency to exalt the market for ideas and
to depreciate the market for goods.'' But why should this differential
treatment be so? For Coase the distinction is invalid as ``{[}t{]}here
is no fundamental difference between these two markets.'' He goes on to
argue that we should use the same approach for all markets when deciding
upon public policy.

For Coase the explanation for the tendency on the part of intellectuals
to make a sharp distinction between the two markets is self-interest and
self-esteem. In offering this explanation, he echoes the Public Choice
literature: self-interest leads intellectuals to see themselves as the
experts who will plan and implement the regulation of the market for
goods, while self-esteem causes them to deny any role for regulation of
their domain - the market for ideas. Coase urges intellectuals to
``\ldots{} adopt a more consistent view.''\footnote{For a book-length
  treatment of such tendencies on the part of intellectuals see
  \citet{Sowell2009}.} Expressing his dismay at this situation he
writes:

\begin{quote}
It is hard to believe that the general public is in a better position to
evaluate competing views on economic and social policy than to choose
between different kinds of food.
\end{quote}

Indeed, this is puzzling, as is the distinction between the ethical
underpinnings of the two markets. The discussion in \citet{Coase1974} is
helpful in framing the current discussion. Most essential is the
perspective that intellectuals operate within a market setting when
producing and promoting their ideas.\footnote{Elsewhere
  \citet{Polanyi2000} has made a similar case.} The sharp distinction
between the two markets when it comes to the role of government
regulation parallels the common sharp distinction of their respective
moral nature. When the scholar operates in the ``market for ideas'' it
is commonly believed that her efforts are inherently ethical, resting on
a long history and strong tradition of common moral values. While it is
often the case, especially among intellectuals, to depreciate the moral
value of capitalist enterprise that is the domain of the financial
trader. Echoing Coase I argue below that this practice is unfounded, and
I urge the adoption of a more consistent view as there is no fundamental
difference between the two activities. They are both examples of
arbitrage within their respective domains.

\section{The Informational Role of
Prices}\label{the-informational-role-of-prices}

Once the domain of intellectuals is seen as a market - the ``market for
ideas'' - we can inquire about the special nature of markets to better
understand their ethical foundation. The essential element is that
competitive markets are mediated by prices, and prices serve as signals
of information. As \citet{Grossman1989} writes ``{[}i{]}t is a common
theme of most discussions of the competitive price system that prices
convey information.''\footnote{See also \citet{Kreps1988} who states
  that ``the notion that prices contain and convey information is
  standard doctrine among economists.''} This concept in its original
formulation is due to \citet{Hayek1945}, who wrote

\begin{quote}
\textbf{\emph{We must look at the price system as \ldots{} a mechanism
for communicating information if we want to understand its real
function}}\ldots{} The most significant fact about this system is the
economy of knowledge with which it operates, or how little the
individual participants need to know in order to be able to take the
right action \ldots{} by a kind of symbol, only the most essential
information is passed on\ldots{}"
\end{quote}

For Hayek, the essential element of markets is that they are information
systems, as we might say in the modern parlance.\footnote{Hayek wrote:
  ``It is more than a metaphor to describe the price system as a kind of
  machinery for registering change, or a \emph{system of
  telecommunications}.'' Emphasis added.} Hayek wrote in the midst of
the so-called socialist calculation debate. The Polish economist Oskar
Lange (see \citet{Lange1936} and \citet{Lange1937}) argued that taking
mathematical models of neoclassical economics seriously, the competitive
allocation of the market could be replicated by a central planner making
socialism a viable economic system. Hayek vehemently opposed this view
highlighting that the would-be central planner faces an insurmountable
knowledge problem because the knowledge that must be made use of for
such planning is dispersed widely among many people, who each possess
only an incomplete and partial knowledge.\footnote{\citet{Polanyi2000}
  makes a similar argument regarding a knowledge problem in science. He
  writes: ``Any attempt to organize the group \ldots{} under a single
  authority would eliminate their independent initiatives, and thus
  reduce their joint effectiveness to that of the single person
  directing them from the centre. It would, in effect, paralyse their
  co-operation.'' His argument regarding science can be extended to
  scholarship in general.}

Similar to Coase, Hayek points out the tendency of some to make a sharp
distinction between different kinds of knowledge. He writes ``one kind
of knowledge, namely, scientific knowledge, occupies now so prominent a
place in public imagination that we tend to forget that it is not the
only kind that is relevant.'' Indeed, by taking advantage of relative
mispricings among commodities the financial trader benefits his fellow
men by allocating resources more efficiently. But this is hardly viewed
as a noble act, not nearly so noble as scientific discovery or academic
expression. Hayek saw keenly the situation:

\begin{quote}
It is a curious fact that this sort of knowledge should today be
generally regarded with a kind of contempt, and that someone who by such
knowledge gains an advantage over somebody better equipped with
theoretical or technical knowledge is thought to have acted almost
disreputably. To gain advantage from better knowledge of facilities of
communication or transport is sometimes regarded as almost dishonest,
although it is quite as important that society make use of the best
opportunities in this respect as in using the latest scientific
discoveries.
\end{quote}

The lesson regarding the relative importance of different kinds of
knowledge that Hayek teaches is a very deep one. Indeed, when this
insight is fully appreciated one begins to see that the use of mundane
kinds of knowledge - Hayek's ``knowledge of time and place'' is
essential. It is not exaggeration to state that if the only knowledge
that society could usefully exploit was that from universities, the
press, literature and science we would all starve to death in very short
order. The financial arbitrageur is a perfect example of Hayek's ``man
on the spot'' and his actions benefit his fellows greatly. Still, when
one thinks of the abstract concept of the pursuit of knowledge, images
of the scientist in the lab or of the scholar in the library come more
easily to mind than the trader scanning prices on his bank of monitors.
But for Hayek:

\begin{quote}
the \emph{arbitrageur} who gains from local differences of commodity
prices, {[}is{]} performing {[}an{]} eminently useful function based on
special knowledge of circumstances of the fleeting moment not known to
others.
\end{quote}

The actions of the arbitrageur are indeed eminently useful! Consider the
commodities that we require in our daily lives such as sugar, coffee,
wheat, milk, and oil that come to us more efficiently and at lower cost
than would otherwise be the case if not for the actions of commodity
traders. In addition, because of financial traders retirement is secured
more soundly through investment. And risks are insured against,
eliminated and transferred through financial markets that we would
otherwise have to bear.

Hayek compared the price system to ``a kind of machinery for registering
change'' and as a ``system of telecommunications.'' Similarly,
\citet{Scarf1990} has made an analogy between the market institution and
a massively distributed analog computer that computes values in the form
of prices. Following this line of reasoning, all market processes are
knowledge processes. Therefore there is no sharp distinction between the
``market for ideas'' and the ``market for goods'', as Coase referred to
them. They are simply both markets and differ only in their relative
kinds of knowledge. \citet{Sowell1996} has made this point quite
succinctly:

\begin{quote}
While market economies are often thought of as money economies, they are
still more so knowledge economies\ldots{} Economic transactions are
purchases and sales of knowledge.
\end{quote}

\begin{quote}
After all, the cavemen had the same natural resources at their disposal
as we have today\ldots{} We are all in the business of buying and
selling knowledge from one another, because we are each so profoundly
ignorant of what it takes to complete the whole process of which we are
part.
\end{quote}

My argument, echoing Coase and augmented by a sound understanding of
market processes as information processes from Hayek, is that we should
adopt a more coherent view of the ethical and moral foundations of both
markets, for they are fundamentally of the same nature.

\section{Arbitrage Choice Theory}\label{arbitrage-choice-theory}

The link between the market for ideas and the market for goods is
further strengthened by considering the work of the decision theorist
Robert Nau. For example, \citet{NauMcCardle1991} state that the
principle of no-arbitrage, which is fundamental in finance theory, is
the common foundation of decision theory, game theory and competitive
market theory. \citet{Nau1999} refers to this as Arbitrage Choice
Theory, following the Bayesian Dutch book arguments of
\citet{deFinetti1937}. A Dutch book is a situation in which an
individual holds incoherent beliefs (beliefs inconsistent with the basic
axioms of probability), and if forced to bet on them exposes themselves
to a guaranteed loss. When market transactions are seen as knowledge
transactions, it becomes clear that a Dutch book is equivalent to
arbitrage - the good old fashioned kind often derided from the world of
finance.

When a trader makes a profit by simultaneously buying and selling
financial securities to exploit relative mispricings, this arbitrage is
simply the exploitation of a Dutch book expressed as incoherent market
prices. Likewise the intellectual pursuit of knowledge, what Coase
called the ``market for ideas'' can be seen as the pursuit of arbitrage,
or Dutch books, in the existing body of academic knowledge. Consider a
young scholar who discovers an error in the received theory of their
specific academic domain, and who makes a name for themselves by
publishing an article correcting the error in a reputable academic
journal. This can now be easily understood to be the exploitation of a
Dutch book, or intellectual arbitrage.

\subsection{The Dutch Book Theorem
Explained}\label{the-dutch-book-theorem-explained}

Coherence in de Finetti's definition of probability is the counterpart
of the modern theory of no-arbitrage in finance.\footnote{See
  \citet{PressaccoZiani2010}.} Interestingly, de Finetti took a very
operational and economic approach to eliciting subjective probabilities.

The axioms of probability which characterize any theory of probability
arise logically from the condition that subjective probabilities
function as prices obeying coherency (i.e.~the arbitrage free
conditions).

Looking at problems of uncertainty as situations in which various
contingent claims are traded is the key insight offered by the
state-preference theory.

The dictionary definition of a Dutch book is the following:

\begin{quote}
``A principle for dynamic decision-making situations that leads to a
sequence of bets finishing in an inescapable loss, as in the case of a
bookmaker always gaining from gambling on the outcome of horse
races.''\footnote{See the Routledge Dictionary of Economics entry on
  ``Dutch books.''}.
\end{quote}

The following example serves to demonstrate the principle. Consider
agent A's degree of belief regarding the flipping of a fair coin. The
event space is represented by \(S = \{H, T\}\) where \(H\) represents
`heads' and \(T\) represents `tails.' Suppose A's degrees of belief are
expressed as probabilities:
\(P(H) = 0.51 \quad \mbox{and} \quad P(T) = 0.51\). That is, A is
willing to pay \(\$0.51\) to receive \(\$1\) in the event of a `heads',
and also \(\$0.51\) to receive \(\$1\) in the event of a `tails.' It is
easy to see that A is acting incoherently. The book maker will be
willing to take both bets simultaneously. Either `heads' or a `tails'
will occur, so if A offers both bets simultaneously she will be
guaranteed to earn \(\$1\). But the total of her bets is \(\$1.02\), so
net of her bets she will lose \(\$0.02\) whatever the state of the
world. A Dutch book has been made against her.

In the nomenclature of finance a Dutch book is an arbitrage and the
person making a Dutch book against an opponent is called an arbitrageur.
For de Finetti the bookmaker whose gain is certain in a given bet is no
different than the trader who earns a guaranteed profit by buying low
and selling high when an asset is mispriced.

For de Finetti probabilities are prices. In modern finance theory prices
are probabilities. In the end it is an equivalence.

Once seen through the lens of subjective probability, it becomes clear
that what applies to mispriced assets also applies to misplaced
probabilities. They are both examples of incoherent beliefs.

\subsection{Examples of Dutch Book
Arguments}\label{examples-of-dutch-book-arguments}

\begin{itemize}
\item
  Culp \& Miller vs.~Pirrong re: whether or not MG was a prudent hedger
\item
  Swinburne vs.~Howson re: whether or not God exists
\item
  The morality of scholarship vs the morality of arbitrage is an example
  of a DBA.
\end{itemize}

\section{The Dutch Book Argument}\label{the-dutch-book-argument}

\section{Conclusion}\label{conclusion}

\newpage
\singlespacing 
\bibliography{master}

\end{document}
