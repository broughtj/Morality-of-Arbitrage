\documentclass[11pt,]{article}
\usepackage[left=1in,top=1in,right=1in,bottom=1in]{geometry}
\newcommand*{\authorfont}{\fontfamily{phv}\selectfont}
\usepackage[]{mathpazo}


  \usepackage[T1]{fontenc}
  \usepackage[utf8]{inputenc}



\usepackage{abstract}
\renewcommand{\abstractname}{}    % clear the title
\renewcommand{\absnamepos}{empty} % originally center

\renewenvironment{abstract}
 {{%
    \setlength{\leftmargin}{0mm}
    \setlength{\rightmargin}{\leftmargin}%
  }%
  \relax}
 {\endlist}

\makeatletter
\def\@maketitle{%
  \newpage
%  \null
%  \vskip 2em%
%  \begin{center}%
  \let \footnote \thanks
    {\fontsize{18}{20}\selectfont\raggedright  \setlength{\parindent}{0pt} \@title \par}%
}
%\fi
\makeatother




\setcounter{secnumdepth}{0}



\title{The Morality of Arbitrage  }



\author{\Large Tyler J. Brough\vspace{0.05in} \newline\normalsize\emph{Utah State University}  }


\date{}

\usepackage{titlesec}

\titleformat*{\section}{\normalsize\bfseries}
\titleformat*{\subsection}{\normalsize\itshape}
\titleformat*{\subsubsection}{\normalsize\itshape}
\titleformat*{\paragraph}{\normalsize\itshape}
\titleformat*{\subparagraph}{\normalsize\itshape}


\usepackage{natbib}
\bibliographystyle{apsr}



\newtheorem{hypothesis}{Hypothesis}
\usepackage{setspace}

\makeatletter
\@ifpackageloaded{hyperref}{}{%
\ifxetex
  \usepackage[setpagesize=false, % page size defined by xetex
              unicode=false, % unicode breaks when used with xetex
              xetex]{hyperref}
\else
  \usepackage[unicode=true]{hyperref}
\fi
}
\@ifpackageloaded{color}{
    \PassOptionsToPackage{usenames,dvipsnames}{color}
}{%
    \usepackage[usenames,dvipsnames]{color}
}
\makeatother
\hypersetup{breaklinks=true,
            bookmarks=true,
            pdfauthor={Tyler J. Brough (Utah State University)},
             pdfkeywords = {arbitrage, Dutch books, morality},  
            pdftitle={The Morality of Arbitrage},
            colorlinks=true,
            citecolor=blue,
            urlcolor=blue,
            linkcolor=magenta,
            pdfborder={0 0 0}}
\urlstyle{same}  % don't use monospace font for urls



\begin{document}
	
% \pagenumbering{arabic}% resets `page` counter to 1 
%
% \maketitle

{% \usefont{T1}{pnc}{m}{n}
\setlength{\parindent}{0pt}
\thispagestyle{plain}
{\fontsize{18}{20}\selectfont\raggedright 
\maketitle  % title \par  

}

{
   \vskip 13.5pt\relax \normalsize\fontsize{11}{12} 
\textbf{\authorfont Tyler J. Brough} \hskip 15pt \emph{\small Utah State University}   

}

}







\begin{abstract}

    \hbox{\vrule height .2pt width 39.14pc}

    \vskip 8.5pt % \small 

\noindent In this paper, I argue that one cannot rationally believe that financial
arbitrage is an inherently immoral pursuit, while simultaneously
believing that the pursuit of academic knowledge is an inherently moral
pursuit. I argue that these two seemingly separate activities are really
one and the same. Further those that do hold such contradictory beliefs
are behaving incoherently and have a Dutch book made against them.


\vskip 8.5pt \noindent \emph{Keywords}: arbitrage, Dutch books, morality \par

    \hbox{\vrule height .2pt width 39.14pc}



\end{abstract}


\vskip 6.5pt

\noindent  \section{Introduction}\label{introduction}

\begin{quote}
{[}T{]}he shipper who earns his living from using otherwise empty or
half-filled journeys of tramp-steamers, or the estate agent whose whole
knowledge is almost exclusively one of temporary opportunities, or the
\emph{arbitrageur} who gains from local differences of commodity prices,
are all performing eminently useful functions based on special knowledge
of circumstances of the fleeting moment not known to others. -- F.A.
Hayek
\end{quote}

In this essay I follow the lead of \citet{Coase1974}, and examine the
very common differential treatment of the market for goods and the
market for ideas.

The plan of this paper is as follows: section 2 introduces Coases's
original argument, and his conclusion that their different treatment is
incongruous; section 3 discusses the view in \citet{Hayek1945} that the
essential nature of the economic problem is one of the utilization of
knowledge. For Hayek, the economic process is an information process and
markets can be seen, using the modern parlance, as information systems.
There is thus no justification for the differential treatment of goods
and ideas; section 4 introduces the concept of subjective probability
following \citet{deFinetti1937}, as well as the Arbitrage Choice Theory
outlined in \citet{Nau1999}; section 5 explains, as essential
background, the Dutch Book Theorem; section 6 gives a few chosen
examples of scholarly debates framed as Dutch Book arguments; section 6
frames Coases's original argument as a Dutch Book argument; section 7
concludes.

\section{The Market for Goods and the Market for
Ideas}\label{the-market-for-goods-and-the-market-for-ideas}

\section{The Use of Knowledge in
Society}\label{the-use-of-knowledge-in-society}

\section{Arbitrage Choice Theory}\label{arbitrage-choice-theory}

\section{The Dutch Book Theorem
Explained}\label{the-dutch-book-theorem-explained}

\subsection{Examples of Dutch Book
Arguments}\label{examples-of-dutch-book-arguments}

\section{Coase's Dutch Book Argument}\label{coases-dutch-book-argument}

\section{Conclusion}\label{conclusion}

\newpage
\singlespacing 
\bibliography{master.bib}

\end{document}
